%このサンプルは過去の先輩が残してくれたものに,一部修正を加えたものです.


\documentclass[a4paper, twoside]{jarticle}
% a4paper : A4のサイズに設定
% twoside : 偶数/奇数ページで異なるレイアウト,bookのデフォルト.

\usepackage{tani_resume} % 卒論用
\usepackage{ascmac} % レイアウトを綺麗にする
\usepackage{comment} % 複数行のコメントを使用できる

% 先輩方が全員記述しているため,記述した,検索しても用途が不明
% --- ここから ---
\alignbeforeskip -5mm
\alignafterskip -5mm
\eqnarraybeforeskip -5mm
\eqnarrayafterskip -5mm
% --- ここまで ---

\jptitle{タイトルを書く} % タイトル
\etitle{} % 空白で構わない
\jpauthor{日大 太郎} % 名前
\eauthor{Nichidai Taro} % 名前(ローマ字)
\course{谷聖一 研究室} % {谷聖一 研究室}で良い
\year{2} % 自分たちの代の年度
%多分これを見る頃には平成じゃなくなっているので誰かtani_reusme.styを変更する必要あり

\abstract{ % 概要
概要を書きましょう.
}

% 右上に最終更新時刻を表示する,バックアップを見返す際に便利
% 先輩方は完成時にはコメントアウトしている
\compheading % 最終更新時刻


\begin{document} % 文書(開始)
\maketitle % タイトルを表示する
\begin{multicols}{2} % 2段落にする(開始)
\setcounter{page}{1} % ページ開始番号

\section{はじめに}
ここには「はじめに」を書きます
\subsection{何か書く}
副題ですね

\section{何か書く}
次の内容を書きます
\subsection{何か書く}
次の内容の副題ですね




% multicols を end したあとに参考文献
\begin{thebibliography}{99}
\end{thebibliography}
\end{multicols} % 2段落にする(終了)

\end{document} % 文書(終了)
